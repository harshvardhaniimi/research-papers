%%%%%%%%%%%%%%%%%%%%%%%%%%%%%%%%%%%%%%%%%%%%%%%%%%%%%%%%%%%%%%%%%%%%%%%%%%%%
%% Author template for INFORMS Journal on Data Science (ijds) [interim solution; new styles under construction]
%% Mirko Janc, Ph.D., INFORMS, mirko.janc@informs.org
%% ver. 0.91, March 2015 - updated November 2020 by Matthew Walls, matthew.walls@informs.org
%% Adapted for rticles by Rob J Hyndman Rob.Hyndman@monash.edu. Dec 2021
%%%%%%%%%%%%%%%%%%%%%%%%%%%%%%%%%%%%%%%%%%%%%%%%%%%%%%%%%%%%%%%%%%%%%%%%%%%%
\documentclass[,msom,nonblindrev]{informs}

\OneAndAHalfSpacedXI
%%\OneAndAHalfSpacedXII % Current default line spacing
%%\DoubleSpacedXII
%%\DoubleSpacedXI

%% BEGIN MY ADDITIONS %%
\usepackage{hyperref}

% tightlist command for lists without linebreak
\providecommand{\tightlist}{%
  \setlength{\itemsep}{0pt}\setlength{\parskip}{0pt}}




%% END MY ADDITIONS %%


% Natbib setup for author-year style
\usepackage{natbib}
 \bibpunct[, ]{(}{)}{,}{a}{}{,}%
 \def\bibfont{\small}%
 \def\bibsep{\smallskipamount}%
 \def\bibhang{24pt}%
 \def\newblock{\ }%
 \def\BIBand{and}%


%% Setup of theorem styles. Outcomment only one.
%% Preferred default is the first option.
\TheoremsNumberedThrough     % Preferred (Theorem 1, Lemma 1, Theorem 2)
%\TheoremsNumberedByChapter  % (Theorem 1.1, Lema 1.1, Theorem 1.2)
\ECRepeatTheorems

%% Setup of the equation numbering system. Outcomment only one.
%% Preferred default is the first option.
\EquationsNumberedThrough    % Default: (1), (2), ...
%\EquationsNumberedBySection % (1.1), (1.2), ...

% For new submissions, leave this number blank.
% For revisions, input the manuscript number assigned by the on-line
% system along with a suffix ".Rx" where x is the revision number.
\MANUSCRIPTNO{}

%%%%%%%%%%%%%%%%
\begin{document}
%%%%%%%%%%%%%%%%

% Outcomment only when entries are known. Otherwise leave as is and
%   default values will be used.
%\setcounter{page}{1}
%\VOLUME{00}%
%\NO{0}%
%\MONTH{Xxxxx}% (month or a similar seasonal id)
%\YEAR{0000}% e.g., 2005
%\FIRSTPAGE{000}%
%\LASTPAGE{000}%
%\SHORTYEAR{00}% shortened year (two-digit)
%\ISSUE{0000} %
%\LONGFIRSTPAGE{0001} %
%\DOI{10.1287/xxxx.0000.0000}%

% Author's names for the running heads
% Sample depending on the number of authors;
\RUNAUTHOR{%
true
}
% \RUNAUTHOR{Jones and Wilson}
% \RUNAUTHOR{Jones, Miller, and Wilson}
% \RUNAUTHOR{Jones et al.} % for four or more authors
% Enter authors following the given pattern:
%\RUNAUTHOR{}

\RUNTITLE{Landslide Effect}

\TITLE{Landslide Effect: How Looking Into The Future Messes Up Inventory
in Practice}

\ARTICLEAUTHORS{%
\AUTHOR{ Harshvardhan}
\AFF{Haslam College of Business, University of
Tennessee, \EMAIL{\href{mailto:harshvar@vols.utk.edu}{\nolinkurl{harshvar@vols.utk.edu}}}}

%
}

\ABSTRACT{In this review, I understand the landslide effect as described
in Neale and Willems (2014) and present my distilled learnings.}

\KEYWORDS{seasonal demand; days of supply; safety stock
targets; industry practice; inventory management}

\maketitle


Many products experience seasonal demand. The seasonal patterns are
typically repeated with some frequency annually. Generally, it is also
observed that the ``in-season'' demand period produces a lot higher
demand than the ``non-season'' demand, sometimes as much as 60-70\% of
the demand. For example, Elmer's Products, a Columbus, Ohio based
company popular for its school glue, experiences higher demand for the
months leading up to the back-to-school season, when its forecasts are
multiplicatively high.

In such a transitory period when the product demand period transitions
from a high demand period to a low demand period, the company
experiences a phenomenon that \citet{neale2014} and Kraft Foods describe
as ``Landslide effect''. The company experiences a bloated inventory as
it moves from a low to high season demand and a severe drop in inventory
(and service levels) moving from high to low season.

The Landslide effect is a result of how companies calculate their safety
stock levels. A forward coverage based Days of Supply (DOS) method of
calculating inventory uses the future expected demand in calculating the
inventory. Theoretically, the textbooks teach the safety stock should be
calculated based on past periods of demand (i.e.~backward coverage)
while the future demand dictates the base stock. This result is
counter-intuitive to practitioners.

\citet{neale2014} present a theoretical analysis of the effect and
present following conclusions.

\begin{enumerate}
\def\labelenumi{\arabic{enumi}.}
\tightlist
\item
  Forward coverage based DOS calculation for safety stock results in
  unusually high inventory levels at the end of low season (i.e.~when
  transitioning from low to high season) and unusually low inventory
  inventory levels when transitioning from high to low season.
\item
  The root cause of this increase (and drop) in inventory levels is the
  common industry practice of using forward coverage ``Days of Supply''
  for calculating the safety stock levels.
\item
  \textbf{Impact of seasonality:} When the product is more seasonal, the
  landslide and reverse-landslide effects are more pronounced.
\item
  \textbf{Impact of Lead Time:} Longer lead times lead to greater
  landslide and reverse-landslide effect.
\item
  \textbf{Impact of Demand Uncertainty:} Landslide effect is more severe
  for the products with higher demand uncertainty than products that
  have lower demand uncertainty (as measured from monthly coefficient of
  variation).
\item
  This effect has a relatively simple fix: use preceding demand for
  calculating the safety stock, while retaining the same character of
  Days of Supply (DOS).
\end{enumerate}

\hypertarget{current-practice-forward-days-of-coverage}{%
\section{Current Practice: Forward Days of
Coverage}\label{current-practice-forward-days-of-coverage}}

The common method for forecasting seasonal demand is a Days of Supply
(DOS) inventory targets. As the name implies, DOS targets express the
desired inventory levels in days (or units of time). This has several
nice properties.

First, DOS targets intuitively self adjusts with demand. As the demand
grows or shrinks, the safety stock changes appropriately. If the
relative uncertainty (measured by coefficient of variation) remains
relatively constant and only the mean fluctuates through the season,
then the safety stock (\(z\sigma\sqrt{T}\)) will be the same for each
season.\footnote{This is simple to prove. Assume \(\sigma_1/\mu_1 = C\)
  for the period of high sales (1) and \(\sigma_2/\mu_2 = C\) for period
  of low sales (2). When safety stock is expressed in DOS, it will both
  be
  \(z \sigma_1 \sqrt{T} / \mu_1 = z \sigma_2 \sqrt{T} / \mu_2 = z C \sqrt{T}\).}
Second, the DOS can itself be looked as a measure of the average time
the safety stock can be used.

\hypertarget{example}{%
\subsection{Example}\label{example}}

\textbf{INSERT: Maybe add an example if time permits.}

\hypertarget{adaptive-base-stock-inventory-policy}{%
\section{Adaptive Base Stock Inventory
Policy}\label{adaptive-base-stock-inventory-policy}}

Calculating inventory requirements for non-stationary (i.e.~seasonal)
products is called adaptive base stock policy. Plans are reviewed
periodically and replenishment orders are calculated to bring the
inventory levels to the target level. The target level, or base stock,
is a function of the demand forecasts, target service levels (or safety
stock targets). Safety stock targets are usually calculated using the
target service levels.

Calculating the loss due to shortages is difficult for practitioners;
however they do have a good estimate of the expected demand fulfillment
levels. This implies that they would rather set the service level
(\(\alpha\)) than prescribe the cost of missing a target.

For the coming proofs, \citet{neale2014} define a few variables. Review
period is one time unit.

\begin{itemize}
\item
  \(T\) is deterministic lead time,
\item
  \(d(a, b)\) is the demand over the time-period \((a, b)\),
\item
  \(\mu(t)\) is the expected (or mean) demand in period \(t\),
\item
  \(\sigma(t)\) is the standard deviation of demand in period \(t\),
\item
  \(\alpha(t)\) is the target service level in period \(t\).
\end{itemize}

At time \$t\$, the order placed in time-period \(t-T\) would have
arrived. At time \$t-T\$, this order brought the adaptive base stock to
level \$B(t - T)\$, which denotes the net of all demand up to and
including time period \$t-T\$. Total demand from the time \(t\) is
\$d(t-T, t)\$. Therefore, the inventory at time \(t\) is

\[
I(t) = B(t-T) - d(t-T, t).
\]

To avoid stockouts, we would need \(I(t) \geq 0\) at all time periods.
This can be used to define the service levels.

\[
P\{d(t-T, t) \leq B(t-T)\} = \alpha(t).
\]

Assuming that the demand is independently, identically and normally
distributed, the base stock at time period \(t-T\) can be calculated as
the sum of all expected cycle stocks and safety stocks.\footnote{What
  this means is easy to verify by expanding the indices. For example,
  \(\sum_{i = 1}^T \mu(t-T+i)\) is actually
  \(\mu(t - (T-1)) + \mu(t - (T-2)) + \mu(t - (T-3)) +...\) and
  \(\sum_{i = 1}^T \sigma^2(t - T +i)\) is
  \(\sigma^2(t - (T -1)) + \sigma^2(t - (T -2)) + \sigma^2(t - (T -3)) + ...\).}

\[
B(t-T) = \sum_{i = 1}^T \mu(t-T+i) + \Phi^{-1}(\alpha(t)) \sqrt{\sum_{i = 1}^T \sigma^2(t - T +i)}.
\]

The safety stock level in the previous equation is the latter part.

\[
SS^N(t) = \Phi^{-1}(\alpha(t)) \sqrt{\sum_{i = 1}^T \sigma^2(t - T +i)}
\]

The key observation from this result is that the variance of demand in
periods \(t - T + 1\) through \(t\) drives the safety stock at time
\(t\). The safety stock is a function of the demand parameters of the
preceding time indexes. This is different from the base stock level
calculations which changes in anticipation of the upcoming demand.

The base stock level triggers a replenishment order that does not result
in a safety stock on hand until \(T\) periods later. We plan for safety
stock in advance but it doesn't get fulfilled until the end of
replenishment lead time, \(T\). This is the source of why getting this
timing right is important.

Practitioners fail to appreciate this fact that for the time period that
the safety stock order is placed, we do not actually know the demand for
that period. In actuality, the safety stock could face depletion during
that period which would render our model not as useful.

This fundamental disconnect serves the base to defining and analyzing
the landslide effect due to a mismatch is timing of supply and demand.

\hypertarget{landslide-effect}{%
\section{Landslide Effect}\label{landslide-effect}}

Consider a product with two seasons, \(j = 1, 2\). Let \(\mu_j\) and
\(\sigma_j\) reflect the mean and standard deviation of demand for those
periods. Let \(r = \mu_1/\mu_1\) represent the mean demand multiple. The
coefficient of variation is \(C = \sigma_1/\mu_1 = \sigma_2/\mu_2\). The
service level (\(\alpha\)) is assumed to be greater than 50\%, else
there would be no need for a safety stock. Lead time is represented as
\(T\).

In this scenario, the DOS safety stock is calculated as
\(z\sigma\sqrt{T}\), where \(z = \Phi^{-1}(\alpha)\). Note that we have
the same DOS target (\(D\)) for both the seasons as
\(D = z\sigma_1\sqrt{T}/\mu_1 = z\sigma_2\sqrt{T}/\mu_2\). For
simplicity, assume \(D\) is an integer.The forward coverage metric
(\(SS^F(t)\) ) would convert this safety stock target to units by
summing up the upcoming forecasts. That is,

\[
SS^F(t) = \sum_{i = 1}^D \mu(t+i).
\]

\hypertarget{comparing-forward-and-backward-coverage}{%
\subsection{Comparing Forward and Backward
Coverage}\label{comparing-forward-and-backward-coverage}}

For this purpose, \citet{neale2014} study the ratio forward and backward
coverage metrics given by the following expression.

\[
\frac{SS^F(t)}{SS^N(t)} = \frac{\sum_{i = 1}^D \mu(t + i)}{z \sqrt{\sum_{i = 1}^T \sigma^2(t-T+i)}}.
\]

This expression can be simplified for three cases.

\hypertarget{case-1-t-leq-s-d-or-tst}{%
\subsubsection{\texorpdfstring{Case 1: \(t \leq s-D\) or
\(t>s+T\)}{Case 1: t \textbackslash leq s-D or t\textgreater s+T}}\label{case-1-t-leq-s-d-or-tst}}

For far ahead or before the transitory period (when we move from high to
low or low to high season), the ratio is exactly equal to 1.

\hypertarget{proof}{%
\paragraph{Proof}\label{proof}}

The numerator in this case would be \(D \times \mu_j\) and the
denominator would be \(z \times T \times \sigma_j\), which when taken a
ratio of would result in 1. Thus, \(s-D+1 \leq t \leq s+T-1\) is the
period of interest for the landslide effect (called transitory window in
the paper).

\hypertarget{case-2-s---d-1-leq-t-leq-s}{%
\subsubsection{\texorpdfstring{Case 2:
\(s - D + 1 \leq t \leq s\)}{Case 2: s - D + 1 \textbackslash leq t \textbackslash leq s}}\label{case-2-s---d-1-leq-t-leq-s}}

In this case, let \(b\) be the time periods before the final season
(\(b = 0, 1, …, D-1\)). Then the ratio becomes:

\[
\frac{SS^F(t)}{SS^N(t)} = \frac{b \mu_1 + (D-b)\mu_2}{z \sqrt{T \sigma^2}}.
\]

Both, numerator and denominator are simplified by opening up the
original formula. Now, substituting \(\mu_2 = r \mu_1\) and
\(D = z\sigma_1\sqrt{T}/\mu_1\), we obtain the following.

\[
\frac{SS^F(t)}{SS^N(t)} = r + \frac{b(1-r)}{D}.
\]

\hypertarget{case-3-s-leq-t-leq-st-1}{%
\subsubsection{\texorpdfstring{Case 3:
\(s \leq t \leq s+T-1\)}{Case 3: s \textbackslash leq t \textbackslash leq s+T-1}}\label{case-3-s-leq-t-leq-st-1}}

Let \(a = t-s\) represent the number of periods after the final period
of season one (\(a = 0, 1, …, s+T-1\)). Then, the ratio becomes the
following upon opening up the summation and adding up the terms.

\[
\frac{SS^F(t)}{SS^N(t)} = \frac{D\mu_2}{z\sqrt{a\sigma_2^2 + (T-a)\sigma_1^2}}
\]

Again by substituting for \(r = \sigma_2/\sigma_1\) and
\(D = z\sigma_2\sqrt{T}/\mu_2\), we obtain the following result.

\[
\frac{SS^F(t)}{SS^N(t)} = \sqrt{\frac{Tr^2}{a(r^2 - 1) + T}}.
\]

\hypertarget{proposition-1}{%
\subsection{Proposition 1}\label{proposition-1}}

For the transition window from \(D-1\) periods before the end of a
season until \(T-1\) periods after the end of that season:

\begin{enumerate}
\def\labelenumi{\arabic{enumi}.}
\item
  If \(r<1\), then \(\frac{SS^F(t)}{SS^N(t)} < 1\)

  When transitioning from high season to low season, the forward
  coverage approach results too little safety stock in each period of
  the transition window. The greatest error (i.e., smallest ratio)
  occurs in the final period of the high season,
  i.e.~\(\frac{SS^F(t)}{SS^N(t)} = r\).
\item
  If \(r>1\), then \(\frac{SS^F(t)}{SS^N(t)} > 1\)

  When transitioning from low season to high season, the forward
  coverage approach results in too much safety stock in each period of
  the transitory window. The greatest error (i.e.~largest ratio) will
  occur in the final period of the low season when
  \(\frac{SS^F(t)}{SS^N(t)} = r\).
\end{enumerate}

\hypertarget{proof-1}{%
\paragraph{Proof}\label{proof-1}}

Consider the case when \(r < 1\) and \(s-D+1 \leq t \leq s\). We can
multiply numerator and denominator by \(D\) to obtain
\(\frac{Dr + b(1-r)}{D}\). As \(0 \leq b < D\) and \(1-r>0\) by
definition, we can see that \(Dr + b(1-r) < Dr + D(1-r) = D\). Thus, the
maximum possible value of the ratio is \(D/D = 1\).

Now, the first derivative of the expression for Case 2 is

\[
\frac{d}{db} \left(r + \frac{b(1-r)}{D}\right) = \frac{1-r}{D} > 0.
\]

This implies that the ratio is increasing. Thus, it's lower bound will
be obtained when \(b = 0\) and then the ratio
\(\frac{SS^F(t)}{SS^N(t)} = 1\).

For the case 3, we can see that \(T(r^2-1) < a(r^2-1)\) since \(a<T\)
and \((r^2-1)<0\). Therefore, \(\frac{Tr^2}{a(r^2 - 1) + T}<1\), which
means \(\frac{SS^F(t)}{SS^N(t)} < 1\). The first derivative of the
equation in Case 2 with respect to \(a\) is

\[
\frac{d}{da} \left( \sqrt{\frac{Tr^2}{a(r^2 - 1) + T}}  \right) = \frac{(1-r^2) \sqrt{Tr^2}}{2(a(r^2 - 1)T)^{3/2}}.
\] Now, \(a(r^2-1) \geq -a\) as \((r^2 - 1)\geq -1\) and \(a \geq 0\).
As \(a<T\), \(-a>-T\) and thus \(a(r^2-1)>-T\). The denominator of the
above derivative is also positive. Thus, the equation is an increasing
function of \(a\) annd obtains the minimum value when \(a = 0\), where
\(\frac{SS^F(t)}{SS^N(t)} = 1\).

\hypertarget{proposition-2}{%
\subsection{Proposition 2}\label{proposition-2}}

The worst case expected service level when transitioning from a high
season to a low season via forward coverage occurs in the final period
of the high season and is equal to
\(\Phi^{-1}(r \times \Phi^{-1}(\alpha))\).

\hypertarget{proof-2}{%
\paragraph{Proof}\label{proof-2}}

Using the service level formula, we see

\[
\alpha^F(t) = P\left[ d(t-T+1, t) \leq SS^F(t) + \sum_{i = 1}^T \mu(t-T+i) \right]\\
= \Phi \left( \frac{SS^F(t)}{\sqrt{\sum_{i = 1}^T} \sigma^2(t-T+i)} \right)\\
= \Phi \left(z \times \frac{SS^F(t)}{SS^N(t)} \right).
\]

In the previous proposition (2), we proved that the ratio
\(\frac{SS^F(t)}{SS^N(t)}\) will attain it's minimum when it is equal to
\(r\), which is when we will have the worst expected service level.
Therefore, the worst expected service level is \(\Phi(z\times r)\) which
is equal to \(\Phi(r \times \Phi^{-1}(\alpha))\) as
\(\frac{SS^F(t)}{SS^N(t)} = F(\alpha)\).

\hypertarget{proposition-3}{%
\subsection{Proposition 3}\label{proposition-3}}

The magnitude of the safety stock and service level errors in the each
period under forward-coverage is increasing in \(r>1\), decreasing in
\(r<1\) and non-decreasing in \(T\) and \(C\).

This proposition helps us understand what drives the landslide effect.
Products with high seasonality, longer lead times and high demand
uncertainty are most susceptible to landslide effect.

\begin{enumerate}
\def\labelenumi{\arabic{enumi}.}
\tightlist
\item
  Longer lead time would make the forward coverage based safety stock to
  look more forward when it should be using the backward days of supply.
\item
  Higher demand certainty or volatility increases the \(\sigma\) in the
  calculation of the safety stock, which means that any misalignment is
  further amplified. (Note that \(\sigma/\mu = C\).)
\item
  The ratio of mean demands (\(r\)) between in-season and out-season
  cycles further intensifies how much we experience the landslide
  effect.
\end{enumerate}

The target service level doesn't have a closed form solution to
represent the relationship with other variables. Therefore, we cannot
make a definitive claim about how a high service level affects other
parameters, or vice-versa.

\hypertarget{proof-3}{%
\paragraph{Proof}\label{proof-3}}

The safety stock error ratio \(\frac{SS^F(t)}{SS^N(t)}\) should ideally
be 1, i.e.~the forward and backward coverage should give us the same
results. Therefore, we should be investigating it's (absolute) deviation
from 1, defined as
\(\epsilon_{SS} = \left| \frac{SSF}{SSN} - 1 \right|\).

\textbf{Case A.} Consider the case when \(r<1\) and
\(s-D+1 \leq t \leq s\), i.e.~the case when we're transitioning from low
demand season to high demand season up to time period \$D\$. From the
first theorem and first proposition, we found that in this case
\(\frac{SSF}{SSN} < 1\). In that case,
\(\epsilon_{SS} = 1-\frac{SSF}{SSN}\).

Using the equation obtained in Case 2 of Theorem 1,

\begin{align}
\epsilon_{SS} &= 1 - \frac{SS^F}{SS^N}\\
&= 1 - r - \frac{b(1-r)}{D}.
\end{align}

Since \(b>0\), we will have

\[
\frac{\delta \epsilon_{SS}}{\delta D} = \frac{b(1-r)}{D^2} >0.
\]

This proves that the error is an increasing function of \(D\) and
therefore in \(C\) and \(T\). (Remember that \(D = zC\sqrt{T}\), by
definition. When \(b = 0\), \(\epsilon_{SS} = 1-r\), i.e.~it doesn't
depend on either \(z, C\) or \(T\).

For \(b \geq 0\), we will have

\[
\frac{\delta \epsilon_{SS}}{\delta r} = -1 + \frac{b}{D} < 0,
\]

as \(b < D\), so the \(\epsilon_{SS}\) is decreasing in \(r\).

\textbf{Case B.} Consider the case when \(r>1\) and
\(s-D+1\leq t \leq s\). In this case, \(\frac{SSF}{SSN}<1\). Using the
equation obtained in Case 2 of Theorem 1, we get
\(\epsilon_{SS} = r + \frac{b(1-r)}{D} - 1\).

For any \(b>0\), we have

\[
\frac{\delta \epsilon_{SS}}{\delta D} = \frac{b(r-1)}{D^2} > 0.
\]

Therefore, the error is increasing with respect to \(D\) and therefore
in \(C\) and \(T\). (Recall that \(D = zC\sqrt{T}\), be definition).

For \(b=0\), the error doesn't depend on \(D\) (i.e.~neither on \(C\)
nor on \$T\$). For \(b \geq 0\),

\[
\frac{\delta \epsilon_{SS}}{\delta r} = 1 + \frac{b}{D} > 0.
\]

Therefore, the error is increasing in \(r\).

\hypertarget{conclusion}{%
\section{Conclusion}\label{conclusion}}

In this paper, I reviewed how companies can calculate their safety
stocks in a more scientific manner and how it can deviate from the
principles. Landslide effect is visible from the example and the case
study presented in the paper along with several companies' experiences.

% Appendix here
% Options are (1) APPENDIX (with or without general title) or
%             (2) APPENDICES (if it has more than one unrelated sections)
% Outcomment the appropriate case if necessary
%
% \begin{APPENDIX}{<Title of the Appendix>}
% \end{APPENDIX}
%
%   or
%
% \begin{APPENDICES}
% \section{<Title of Section A>}
% \section{<Title of Section B>}
% etc
% \end{APPENDICES}


% Acknowledgments here
\ACKNOWLEDGMENT{}

\bibliographystyle{informs2014}
\bibliography{references.bib}



\end{document}
